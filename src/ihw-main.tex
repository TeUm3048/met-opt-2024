\documentclass{amsart}

\usepackage[utf8]{inputenc}
\usepackage[T2A]{fontenc}
\usepackage[english,russian]{babel}
\usepackage{amsthm,amsmath,amsfonts,amssymb}
\usepackage{fullpage}
\usepackage{eufrak}
\usepackage{bbm}

%%% Дополнительная работа с математикой
\usepackage{amsfonts,amssymb,amsthm,mathtools} % AMS
\usepackage{amsmath}
\usepackage{icomma}

%% Шрифты
\usepackage{euscript}	% Шрифт Евклид
\usepackage{mathrsfs}	% Красивый матшрифт

%% Python
\usepackage{pythontex} 

\usepackage[italicdiff]{physics}



%% Свои команды
\DeclareMathOperator{\lb}{\mathop{lb}}	% логарифм по основанию 2
\DeclareMathOperator{\sgn}{\mathop{sgn}}	% сигнум
\renewcommand{\Im}{\mathop{\mathrm{Im}}\nolimits}	% мнимая часть
\renewcommand{\Re}{\mathop{\mathrm{Re}}\nolimits}	% вещественная часть
\renewcommand{\emptyset}{\varnothing}	% пустое множество
\renewcommand{\le}{\leqslant}	% отечественная версия "меньше или равно"
\renewcommand{\ge}{\geqslant}	% отечественная версия "больше или равно"
\renewcommand{\epsilon}{\varepsilon}	% стандартная "эпсилон"
\renewcommand{\phi}{\varphi}	% стандартная "фи"
\newcommand{\const}{\mathrm{const}}	% константа
\newcommand{\transpose}{^{\mathsf{T}}} % транспонирование

%% Множества чисел
\DeclareMathOperator{\Natural}{\mathbb{N}}	% Натуральные числа
\DeclareMathOperator{\Integer}{\mathbb{Z}}	% Целые числа
\DeclareMathOperator{\Integerp}{\mathbb{Z}_{+}}	% Целые неотрицательные числа
\DeclareMathOperator{\Rational}{\mathbb{Q}}	% Рациональные числа
\DeclareMathOperator{\Real}{\mathbb{R}}	% Вещественные числа
\DeclareMathOperator{\Realp}{\mathbb{R}_{>0}}	% Вещественные положительные числа
\DeclareMathOperator{\Realn}{\mathbb{R}_{<0}}	% Вещественные отрицательные числа
\DeclareMathOperator{\Realnn}{\mathbb{R}_{\ge 0}}	% Вещественные неотрицательные числа
\DeclareMathOperator{\Realnp}{\mathbb{R}_{\le 0}}	% Вещественные неположительные числа
\DeclareMathOperator{\Complex}{\mathbb{C}}	% Комплексные числа

%% Заглавные греческие буквы
\DeclareMathOperator{\Alpha}{\mathrm{A}}	% Альфа
\DeclareMathOperator{\Beta}{\mathrm{B}}	% Вета
\DeclareMathOperator{\Epsilon}{\mathrm{E}}	% Эпсилон
\DeclareMathOperator{\Zeta}{\mathrm{Z}}	% Дзета
\DeclareMathOperator{\Eta}{\mathrm{H}}	% Эта
\DeclareMathOperator{\Iota}{\mathrm{I}}	% Йота
\DeclareMathOperator{\Kappa}{\mathrm{K}}	% Каппа
\DeclareMathOperator{\Mu}{\mathrm{M}}	% Мю
\DeclareMathOperator{\Nu}{\mathrm{N}}	% Ню
\DeclareMathOperator{\Omicron}{\mathrm{O}}	% Омикрон
\DeclareMathOperator{\Rho}{\mathrm{P}}	% Ро
\DeclareMathOperator{\Tau}{\mathrm{T}}	% Тау
\DeclareMathOperator{\Chi}{\mathrm{X}}	% Хи

%% Теория вероятностей
\renewcommand{\Prob}{\mathbb P}	% вероятность
\newcommand{\Expect}{\mathbb E}	% математическое ожидание
\renewcommand{\Variance}{\mathbb D}	% дисперсия
\newcommand{\Entropy}{\mathbb H}	% энтропия
\DeclareMathOperator{\cov}{\mathop{cov}}	% ковариация
\DeclareMathOperator{\supp}{\mathop{supp}}	% носитель
\DeclareMathOperator{\Skewness}{\mathop{Skew}}	% коэффициент асимметрии
\DeclareMathOperator{\Kurtosis}{\mathop{Kurt}}	% коэффициент эксцесса

%%% Статистический анализ
\newcommand*{\moment}[1]{\overline{#1}}	% выборочный момент
\DeclareMathOperator{\hskew}{\mathop{\widehat{Skew}}}	% выборочный коэффициент асимметрии
\DeclareMathOperator{\hkurt}{\mathop{\widehat{Kurt}}}	% выборочный коэффициент эксцесса
%% Однопараметрические распределения
\newcommand*{\chisq}[1]{\chi^2_{#1}}	% Распределение хи-квадрат
\newcommand*{\Stud}[1]{\mathcal{S}_{#1}}	% Распределение Стьюдента
\newcommand*{\Exp}[1]{\mathop{\mathrm{Exp}}(#1)}	% Показательное распределение
\newcommand*{\Bern}[1]{\mathop{\mathrm{Bern}}(#1)}	% Распределение Бернулли
\newcommand*{\Geom}[1]{\mathop{\mathrm{Geom}}(#1)}	% Геометрическое распределение
\newcommand*{\Pois}[1]{\mathop{\mathrm{Pois}}(#1)}	% Распределение Пуассона
%% Двухпараметрические распределения
\newcommand*{\FS}[2]{\mathcal{F}_{#1, #2}}	% Распределение Фишера-Снедекора
\newcommand*{\Norm}[2]{\mathcal{N}(#1, #2)}	% Нормальное распределение
\newcommand*{\Unif}[2]{\mathcal{U}(#1, #2)}	% Равномерное распределение
\newcommand*{\DE}[2]{\mathop{\mathrm{DE}}(#1, #2)}	% Распределение Лапласа
\newcommand*{\Cauchy}[2]{\mathop{\mathrm{C}}(#1, #2)}	% Распределение Коши
\newcommand*{\Binom}[2]{\mathop{\mathrm{Binom}}(#1, #2)}	% Биномиальное распределение
\newcommand*{\Betadist}[2]{\mathop{\mathrm{Beta}}(#1, #2)}	% Бета-распределение
\newcommand*{\Gammadist}[2]{\mathop{\mathrm{Gamma}}(#1, #2)}	% Гамма-распределение
%% Ажурные и готические буквы
\newcommand*{\Acl}{\mathcal{A}}	% A красивое
\newcommand*{\Ccl}{\mathcal{C}}	% C красивое
\newcommand*{\Fcl}{\mathcal{F}}	% F красивое
\newcommand*{\Icl}{\mathcal{I}}	% I красивое
\newcommand*{\Kcl}{\mathcal{K}}	% K красивое
\newcommand*{\Pcl}{\mathcal{P}}	% P красивое
\newcommand*{\Ycl}{\mathcal{Y}}	% Y красивое
\newcommand*{\Afr}{\mathfrak{A}}	% A готическое
\newcommand*{\Bfr}{\mathfrak{B}}	% B готическое
\newcommand*{\Ffr}{\mathfrak{F}}	% F готическое
\newcommand*{\Kfr}{\mathfrak{K}}	% K готическое
\newcommand*{\Xfr}{\mathfrak{X}}	% X готическое
%% Теория оценивания
\newcommand*{\ind}[1]{\mathbbm{1}_{\lbrace #1 \rbrace}}	% индикаторная функция
\newcommand*{\bias}[2]{\mathop{\mathrm{bias}}\nolimits_{#1}(#2)}	% смещение

%% Перенос знаков в формулах (по Львовскому)
\newcommand*{\hm}[1]{#1\nobreak\discretionary{}
	{\hbox{$\mathsurround=0pt #1$}}{}}

%%% Работа с картинками
\usepackage{graphicx,xcolor}	% Для вставки рисунков
\graphicspath{{images/}{images2/}}	% папки с картинками
\setlength\fboxsep{3pt}	% Отступ рамки \fbox{} от рисунка
\setlength\fboxrule{1pt}	% Толщина линий рамки \fbox{}
\usepackage{wrapfig}	% Обтекание рисунков и таблиц текстом
\RequirePackage{caption}
\DeclareCaptionLabelSeparator{defffis}{ --- }
\captionsetup{justification=centering,labelsep=defffis}
\usepackage{float}
\usepackage{tikz}
\usepackage{pgfplots}
\pgfplotsset{compat=newest}
\usetikzlibrary{patterns}
\usetikzlibrary{calc}
\usepgfplotslibrary{fillbetween}
\usepackage{svg}


%%% Работа с таблицами
\usepackage{array,tabularx,tabulary,booktabs}	% Дополнительная работа с таблицами
\usepackage{longtable}	% Длинные таблицы
\usepackage{multirow}	% Слияние строк в таблице
\usepackage{makecell}
\usepackage{multicol}
\usepackage{diagbox}


\renewcommand{\qedsymbol}{$\blacksquare$}
\renewcommand{\innerproduct}[1]{\left\langle #1, #1 \right\rangle}
\renewcommand{\innerproduct}[2]{\left\langle #1, #2 \right\rangle}
\newcommand{\dprod}{\innerproduct}

\DeclareMathOperator*{\argmax}{arg\,max}
\DeclareMathOperator*{\argmin}{arg\,min}

\newtheorem{problem}{Задание}  \setcounter{problem}{6}

\begin{document}
\newcommand{\problemset}[1]{
	\begin{center}
		\Large #1
	\end{center}
}

\begin{tabbing}
	\hspace{11cm} \= Студент: \= Рыжиков Иван \\	% не забудьте исправить, студент Вы или студентка :)
	% (а то некоторые забывают)
	\> Группа: \> 2381 \\	% Здесь меняете № группы
	% \> Вариант: \> 18 \\		% А здесь меняете № варианта
	\> Дата: \> \today		% А вот здесь ничего не меняем!!!
\end{tabbing}
\hrule
\vspace{1cm}	% в данном файле меняем только Пол, Фамилию Имя, № группы и № варианта
\problemset{Методы оптимизации}
% \problemset{Индивидуальное домашнее задание №4}	% поменяйте номер ИДЗ

\renewcommand*{\proofname}{Доказательство}

Прим. во всех задачах используется обозначение $\grad{f(\vb x)}$ — градиент функции $f(\vb x)$,
и $\dprod{\vb x}{\vb y}$ — скалярное произведение $\vb x$ и $\vb y$.

\begin{problem}
Зная формулу Ньютона-Лейбница:

\[
    \int_{a}^{b} f(x) \, dx = F(b) - F(a)
\]

где $F(x)$ - первообразная функции $f(x)$, непрерывной на отрезке $[a, b]$.

Доказать, что:

\[
    g(\vb{x} + \vb{y}) = g(\vb{x}) + \int_{0}^{1} \dprod{ \nabla g(\vb{x} + t \vb{y}) }{\vb{y} }\, dt
\]
\end{problem}


\begin{proof}
    Рассмотрим функцию $g(\vb{x} + t \vb{y})$ зависящую от параметра $t$.
    \begin{itemize}
        \item Продифференцируем её по $t$ по правилу дифференцирования сложной функции (chain rule):
              \[
                  \dv{t}(g(\vb{x} + t \vb{y})) = \dprod{ \grad{ g(\vb{x} + t \vb{y})} }{\vb{y}}
              \]

        \item Проинтегрируем обе части уравнения по $t$ от $0$ до $1$:
              \[
                  \int_{0}^{1} \dv{t} (g(\vb{x} + t \vb{y})) \, dt
                  = \int_{0}^{1} \dprod{ \grad {g(\vb{x} + t \vb{y}) }}{\vb{y}} \, dt
              \]
        \item По формуле Ньютона-Лейбница левая часть равна:
              \[
                  \int_{0}^{1} \dv{t} (g(\vb{x} + t \vb{y})) \, dt
                  = g(\vb{x} + \vb{y}) - g(\vb{x})
              \]
        \item Получаем:
              \[
                  g(\vb{x} + \vb{y}) - g(\vb{x})
                  = \int_{0}^{1} \dprod{\grad{g(\vb{x} + t \vb{y})}}  {\vb{y}} \, dt
              \]
        \item Переносим $g(\vb{x})$ в правую часть:
              \[
                  g(\vb{x} + \vb{y})
                  = g(\vb{x}) + \int_{0}^{1} \dprod{\grad{g(\vb{x} + t \vb{y})} } {\vb{y}} \, dt
              \]

    \end{itemize}
\end{proof}

\begin{problem}
Для метода наискорейшего спуска доказать, что:
\[
    \dv{t}(f(\vb{x} - t \grad{f(\vb{x})}))|_{t=t_k}
    = \dprod{\grad{f(\vb{x}_{k+1})} }{\grad{f(\vb{x}_k)}}
\]
\end{problem}

\begin{proof}

    Метод наискорейшего спуска заключается в минимизации функции $f\left(\vb{x}\right)$ по направлению антиградиента. Точка минимума функции $f\left(\vb{x}\right)$ находится по формуле:
    \[
        \vb{x}_{k+1} = \vb{x}_k - \alpha_k \grad{f\left(\vb{x}_k\right)}
    \]
    где $\alpha_k$ — шаг метода вдоль антиградиента и $\grad{f\left(\vb{x}_k\right)}$ — градиент функции $f$ в точке $\vb{x}_k$.

    \begin{itemize}
        \item Введем функцию
              \[
                  \vb{x}\left(t\right) = \vb{x}_k - t \grad{f\left(\vb{x}_k\right)}
              \]
              как некоторую кривую, проходящую через точку $\vb{x}_k$
              в направлении антиградиента на шаг $t$.
        \item Далее найдем производную функции $f\left(\vb{x}\left(t\right)\right)$ по $t$, используя правило дифференцирования сложной функции (chain rule):
              \[
                  \dv{t} f\left(\vb{x}\left(t\right)\right) = \dprod{\grad{f\left(\vb{x}\left(t\right)\right)}}{\dv{t} \vb{x}\left(t\right)}
              \]
        \item Так как $\dv{t}\left(\vb{x}\left(t\right)\right) = -\grad{f \left(\vb{x}_k\right)}$,
              то производная функции $f\left(\vb{x}\left(t\right)\right)$ по $t$ равна:
              \[
                  \dv{t} f\left(\vb{x}\left(t\right)\right) = -\dprod{\grad{f\left(\vb{x}\left(t\right)\right)}}{\grad{f\left(\vb{x}_k\right)}}
              \]
        \item Подставим $t= t_k$:
              \[
                  \eval{\dv{t} f\left(\vb{x}_k - t \grad{f\left(\vb{x}_k\right)}\right)}_{t=t_k}
                  = -\dprod{\grad{f\left(\vb{x}_{k+1}\right)}}{\grad{f\left(\vb{x}_k\right)}}
              \]

    \end{itemize}
    Таким образом, доказано, что производная функции $f\left(\vb{x}\left(t\right)\right)$ по $t$ в точке $t_k$ равна скалярному произведению градиентов функции $f$ в точках $\vb{x}_{k+1}$ и $\vb{x}_k$.
\end{proof}

\begin{problem}
Пусть $f(\vb{x}) = 0.5 \vb{x}\transpose A \vb{x} + \dprod{\vb{b}}{\vb{x}}$ — квадратичная функция.

Доказать, что для метода наискорейшего спуска справедливо соотношение

\[
    t_k = \frac{\norm{\grad {f(x_k)}}}{\dprod{\grad{f(x_k)}}{ A \grad{ f(x_k)}}}
\]
\end{problem}

\begin{proof}
    \
    \begin{itemize}
        \item Градиент функции $f(\vb{x})$ равен:
              \[
                  \grad{f(\vb{x})} = A\vb{x} + \vb{b}
              \]
        \item На каждом шаге метода наискорейшего спуска выбирается такой шаг $t_k$, чтобы минимизировать функцию $f(\vb{x}_{k+1})$:
              \[
                  \vb{x}_{k+1} = \vb{x}_k - t_k \grad{f(\vb{x}_k)}
              \]
              Найдем такое $t_k$, при котором функция $f(\vb{x}_{k+1})$ минимальна:
              \[
                  \argmin_{t_k} f(\vb{x}_{k+1})
                  = \argmin_{t_k} f(\vb{x}_k - t_k \grad{f(\vb{x}_k)})
              \]
        \item Подставим $\vb{x}_{k+1}$ в функцию $f(\vb{x})$:
              \[
                  \begin{aligned}
                      f(\vb{x}_{k+1})
                       & = 0.5 (\vb{x}_k - t_k \grad{f(\vb{x}_k)})\transpose A (\vb{x}_k - t_k \grad{f(\vb{x}_k)})
                      + \vb{b}\transpose(\vb{x}_k - t_k \grad{f(\vb{x}_k)})                                        \\
                       & = 0.5 \vb{x}_k\transpose A \vb{x}_k
                      - t_k \grad{f(\vb{x}_k)}\transpose A \vb{x}_k
                      + 0.5 t_k^2 \grad{f(\vb{x}_k)}\transpose A \grad{f(\vb{x}_k)}
                      + \vb{b}\transpose \vb{x}_k
                      - t_k \vb{b}\transpose \grad{f(\vb{x}_k)}
                  \end{aligned}
              \]
        \item Найдем производную функции $f(x_{k+1})$ по $t_k$:
              \[
                  \dv{t_k} f(x_{k+1})
                  = - \grad{f(\vb{x}_k)}\transpose A \vb{x}_k
                  + t_k \grad{f(\vb{x}_k)}\transpose A \grad{f(\vb{x}_k)}
                  - \vb{b}\transpose \grad{f(\vb{x}_k)}
              \]
        \item Сгруппируем слагаемые:
              \[
                  \dv{t_k} f(x_{k+1})
                  = t_k \grad{f(\vb{x}_k)}\transpose A \grad{f(\vb{x}_k)}
                  - \left(\grad{f(\vb{x}_k)}\transpose A \vb{x}_k
                  + \vb{b}\transpose \grad{f(\vb{x}_k)}\right)
              \]
        \item Воспользуемся тем, что $\vb{b}\transpose \grad{f(\vb{x}_k)} = \grad{f(\vb{x}_k)}\transpose \vb{b}$:
              \[
                  \begin{aligned}
                      \dv{t_k} f(x_{k+1})
                       & =  t_k \grad{f(\vb{x}_k)}\transpose A \grad{f(\vb{x}_k)}
                      - \left(\grad{f(\vb{x}_k)}\transpose A \vb{x}_k
                      + \grad{f(\vb{x}_k)}\transpose \vb{b}\right)
                      \\
                       & =  t_k \grad{f(\vb{x}_k)}\transpose A \grad{f(\vb{x}_k)}
                      - \grad{f(\vb{x}_k)}\transpose\left( A \vb{x}_k
                      +  \vb{b}\right)
                      \\
                  \end{aligned}
              \]
        \item Подставим вместо $A\vb{x}_k + \vb{b}$ градиент функции $f(\vb{x}_k)$:
              \[
                  \dv{t_k} f(x_{k+1})
                  = t_k \grad{f(\vb{x}_k)}\transpose A \grad{f(\vb{x}_k)}
                  - \grad{f(\vb{x}_k)}\transpose \grad{f(\vb{x}_k)}
              \]
        \item Воспользуемся услоивем оптимальности шага $t_k$:
              \[
                  \dv{t_k} f(x_{k+1}) = 0
              \]
              Получаем:
              \[
                  t_k \grad{f(\vb{x}_k)}\transpose A \grad{f(\vb{x}_k)}
                  = \grad{f(\vb{x}_k)}\transpose \grad{f(\vb{x}_k)}
              \]
        \item Откуда получаем формулу для шага $t_k$:
              \[
                  t_k
                  = \frac{\grad{f(\vb{x}_k)}\transpose \grad{f(\vb{x}_k)}}{\grad{f(\vb{x}_k)}\transpose A \grad{f(\vb{x}_k)}}
                  = \frac{\norm{\grad{f(\vb{x}_k)}}^2}{\dprod{\grad{f(\vb{x}_k)}}{A \grad{f(\vb{x}_k)} }}
              \]
    \end{itemize}
    Итого, доказано, что для метода наискорейшего спуска справедливо соотношение
    \[
        t_k = \frac{\norm{\grad {f(x_k)}}}{\dprod{\grad{f(x_k)}}{ A \grad{ f(x_k)} }}
    \]
\end{proof}

\renewcommand*{\proofname}{Решение}


\begin{problem}
Найти минимум функции $f(x_1, x_2) = x_1 x_2$ при ограничениях $x_1^2 + x_2^2 \le 1$,
используя теорему Каруша-Джона.
\end{problem}

\begin{proof}
    \
    \begin{itemize}
        \item Запишем функцию Лагранжа:
              \[
                  L(x_1, x_2, \lambda) = x_1 x_2 + \lambda (x_1^2 + x_2^2 - 1)
              \]
        \item Найдем градиент функции Лагранжа:
              \[
                  \grad{L} = \begin{pmatrix}
                      x_2 + 2 \lambda x_1 \\
                      x_1 + 2 \lambda x_2 \\
                      x_1^2 + x_2^2 - 1
                  \end{pmatrix}
              \]
        \item По условию оптимальности из теоремы Каруша-Джона градиент функции Лагранжа равен нулю:
              \[
                  \begin{cases}
                      x_2 + 2 \lambda x_1 = 0 \\
                      x_1 + 2 \lambda x_2 = 0 \\
                      x_1^2 + x_2^2 - 1 = 0
                  \end{cases}
              \]
        \item Решим систему уравнений:
              \[
                  \begin{cases}
                      x_2 + 2 \lambda x_1 = 0 \\
                      x_1 + 2 \lambda x_2 = 0 \\
                      x_1^2 + x_2^2 - 1 = 0
                  \end{cases}
              \]
              Получаем:
              \[
                  \begin{cases}
                      x_2 + 2 \lambda x_1 = 0 \\
                      x_1 + 2 \lambda x_2 = 0 \\
                      x_1^2 + x_2^2 - 1 = 0
                  \end{cases}
                  \Rightarrow
                  \left[
                  \begin{array}{ll}
                      \begin{cases}
                          \lambda = -\frac{1}{2} \\
                          x_1 = x_2 = \pm \frac{1}{\sqrt{2}}
                      \end{cases}
                      \\
                      \\
                      \begin{cases}
                          \lambda = \frac{1}{2} \\
                          x_1 = -x_2 = \pm \frac{1}{\sqrt{2}}
                      \end{cases}
                  \end{array}
                  \right.
              \]
        \item Подставим найденные значения $x_1$ и $x_2$ в функцию $f(x_1, x_2) = x_1 x_2$:
              \[
                  f\left(\frac{1}{\sqrt{2}}, \frac{1}{\sqrt{2}}\right) = \frac{1}{2}
              \]
              \[
                  f\left(\frac{1}{\sqrt{2}}, -\frac{1}{\sqrt{2}}\right) = -\frac{1}{2}
              \]
              \[
                  f\left(-\frac{1}{\sqrt{2}}, \frac{1}{\sqrt{2}}\right) = -\frac{1}{2}
              \]
              \[
                  f\left(-\frac{1}{\sqrt{2}}, -\frac{1}{\sqrt{2}}\right) = \frac{1}{2}
              \]
    \end{itemize}
    Таким образом, минимум функции $f(x_1, x_2) = x_1 x_2$ при ограничениях $x_1^2 + x_2^2 \le 1$ равен $-\frac{1}{2}$
    и достигается в точках $x_1 = - x_2 = \pm \frac{1}{\sqrt{2}}$.
\end{proof}

\begin{problem}
Пусть $f(y, y')$ не зависит от $x$.

Чему равно значение $\dv{x}(f - f_{y'}y')$?
\end{problem}

\begin{proof}
    Введем обозначение $f = f(y, y')$, $f_{y} = \dv{y} f$, $f_{y'} = \dv{y'} f$.

    \begin{itemize}
        \item Продифференцируем $f - f_{y'}y'$ по $x$:
              \[
                  \dv{x}(f - f_{y'}y')
                  = \dv{x} f - \dv{x} (f_{y'}y')
              \]
        \item Так как $f$ не зависит от $x$, то $\dv{x} f = 0$:
              \[
                  \dv{x}(f - f_{y'}y') = - \dv{x} (f_{y'}y')
              \]
        \item Применим правило дифференцирования произведения:
              \[
                  \dv{x} (f_{y'}y') = -f_{y'} \dv{x} (y') - \dv{x} (f_{y'}) y'
              \]
        \item Так как $f$ не зависит от $x$, то $\dv{x} (f_{y'}) = 0$:
              \[
                  \dv{x} (f_{y'}y') = -f_{y'} \dv{x} (y') = -f_{y'} y''
              \]
    \end{itemize}
    Итого, значение $\dv{x}(f - f_{y'}y')$ равно $-f_{y'} y''$.
\end{proof}

\begin{problem}
Составить уравнение Эйлера-Лагранжа для $F = x^2 y + y'^2 x$.
\end{problem}

\begin{proof}
    Формула Эйлера-Лагранжа имеет вид:
    \[
        \dv{x} ( \pdv{F}{y'} ) - \pdv{F}{y} = 0
    \]
    \begin{itemize}
        \item Найдем частную производную $F$ по $y'$:
              \[
                  \pdv{F}{y'} = 2 x y'
              \]
        \item Найдем производную $\pdv{F}{y'}$ по $x$:
              \[
                  \dv{x} ( \pdv{F}{y'} ) = 2 y' + 2 x y''
              \]
        \item Найдем частную производную $F$ по $y$:
              \[
                  \pdv{F}{y} = x^2
              \]
        \item Подставим найденные значения в уравнение Эйлера-Лагранжа:
              \[
                  2 y' + 2 x y'' - x^2 = 0
              \]
    \end{itemize}
    Таким образом, уравнение Эйлера-Лагранжа для $F = x^2 y + y'^2 x$ имеет вид $2 y' + 2 x y'' - x^2 = 0$.
\end{proof}	% для удобства создаём по аналогии файлы ihw1.tex, ihw2.tex, etc
% и просто меняем имя при компиляции
\end{document}